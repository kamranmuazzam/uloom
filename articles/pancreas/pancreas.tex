\documentclass[11pt, a4paper]{article}
\usepackage[utf8]{inputenc}
\usepackage[T1]{fontenc}
\usepackage[margin=1in]{geometry}
\usepackage{titlesec}
\usepackage{enumitem}
\usepackage{amsfonts} % For \mathbb
\usepackage{hyperref}
\hypersetup{
    colorlinks=true,
    linkcolor=blue,
    filecolor=magenta,      
    urlcolor=cyan,
}

% Formatting for sections
\titleformat{\section}{\large\bfseries}{\thesection}{1em}{}
\titleformat{\subsection}{\bfseries}{\thesubsection}{1em}{}
\setlist[itemize]{noitemsep, topsep=0pt} % Removes extra space between list items

\begin{document}

% Title Page
\begin{titlepage}
    \centering
    {\Huge \bfseries Anatomy of the Pancreas}\\[1.5cm]
    {\Large Compiled Notes}\\[2cm]
    \vfill
    {\large \today}
\end{titlepage}

% Table of Contents
\tableofcontents
\pagebreak

\section{Introduction}
\begin{itemize}
    \item Accessory organ of the digestive system.
    \item \textbf{Eponym}: \textit{pan + kreas} (all + flesh).
    \item The pancreas is a \textbf{mixed gland} having both exocrine and endocrine parts.
    \begin{itemize}
        \item The \textbf{exocrine} part secretes \textbf{pancreatic juice} comprising bicarbonate, amylase, lipase, trypsinogen, chymotrypsinogen, and nuclease.
        \item The \textbf{endocrine} part secretes hormones like insulin, glucagon, somatostatin, and pancreatic polypeptide.
    \end{itemize}
    \item In terms of peritoneal relation, it is a \textbf{retroperitoneal} organ except for its \textbf{tail}, which is \textbf{intraperitoneal}.
    \item It lies in the \textbf{epigastrium} and \textbf{left hypochondrium}, while a part also lies in the \textbf{umbilical region}.
\end{itemize}

\section{Macro-Anatomy}
It comprises of a \textbf{head}, \textbf{neck}, \textbf{body}, and \textbf{tail}.

\subsection{Head}
\begin{itemize}
    \item This is the enlarged right end within the C-shaped concavity of the duodenum.
    \item It has superior, right, inferior, anterior, and posterior surfaces.
    \item It has important relations:
    \begin{itemize}
        \item \textbf{Superiorly}: with the first part of the duodenum and the gastroduodenal arteria.
        \item \textbf{To the right}: the second part of the duodenum.
        \item \textbf{Inferiorly}: the third part of the duodenum.
        \item \textbf{Anteriorly}: the transverse colon, few coils of jejunum. Its superior part also relates to the first part of the duodenum and the gastroduodenal artery.
        \item \textbf{Posteriorly}: the bile duct, inferior vena cava, and the diaphragm.
        \item \textbf{In front of the uncinate process}: the superior mesenteric vessels.
    \end{itemize}
\end{itemize}

\subsection{Neck}
\begin{itemize}
    \item It is the most constricted part, which connects the head and the body of the pancreas.
    \item It has anterior and posterior surfaces, with upper and lower borders.
    \item Behind its posterior surface, the \textbf{portal vein} is formed by the union of the superior mesenteric and splenic veins at the level of L2.
\end{itemize}

\subsection{Body}
\begin{itemize}
    \item This is the elongated portion extending from the neck to the tail.
    \item It is triangular in cross-section.
    \item \textbf{Borders}:
    \begin{itemize}
        \item \textbf{Anterior border}: Serves as an attachment for the root of the transverse mesocolon.
        \item \textbf{Superior border}: Has a projection called the \textit{tuber omentale}. It is related to the coeliac trunk above; the splenic artery runs to the left and the hepatic artery to the right.
        \item \textbf{Inferior border}: Related to the superior mesenteric vessels.
    \end{itemize}
    \item \textbf{Surfaces}:
    \begin{itemize}
        \item \textbf{Anterior surface}: Covered by visceral peritoneum.
        \item \textbf{Posterior surface}: Has a groove for the splenic vein.
    \end{itemize}
\end{itemize}

\subsection{Tail}
\begin{itemize}
    \item \textbf{Level}: T-12.
    \item It is the left end of the gland and doesn't have a clear demarcation.
    \item It is \textbf{intraperitoneal} and lies in the \textbf{lienorenal ligament}, which connects it to the spleen.
\end{itemize}

\end{document}