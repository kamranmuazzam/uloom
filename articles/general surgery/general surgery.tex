\documentclass[11pt, a4paper]{article}
\usepackage[utf8]{inputenc}
\usepackage[T1]{fontenc}
\usepackage[margin=1in]{geometry}
\usepackage{titlesec}
\usepackage{enumitem}
\usepackage{manyfoot}
\usepackage{amsfonts} % For \mathbb
\usepackage{hyperref}

\usepackage{imakeidx}
\makeindex


\hypertarget{idx:viva}{}
\hypertarget{idx:written}{}
% Smart commands that automatically create index entries
\newcommand{\viva}{%
    \textsuperscript{\hyperref[idx:viva]{\textbf{v}}}%
    \index{viva@\textbf{v} (viva examination)}%
}
\newcommand{\written}{%
    \textsuperscript{\hyperref[idx:written]{\textbf{w}}}%
    \index{written@\textbf{w} (written examination)}%
}

\hypersetup{
    colorlinks=true,
    linkcolor=blue,
    filecolor=magenta,      
    urlcolor=cyan,
}

\DeclareNewFootnote{A}


% Formatting for sections
\titleformat{\section}{\large\bfseries}{\thesection}{1em}{}
\titleformat{\subsection}{\bfseries}{\thesubsection}{1em}{}
\setlist[itemize]{noitemsep, topsep=0pt} % Removes extra space between list items

\begin{document}



% Title Page
\begin{titlepage}
    \centering
    {\Huge \bfseries General Surgery}\\[1.5cm]
    {\Large Surgery}\\[2cm]
    \vfill
    {\large \today}
\end{titlepage}

% Create target anchors for hyperlinks


\tableofcontents
\pagebreak


\section{Transfusion}

\subsection{Common Blood Products}
The commonly transfused blood products are mentioned below\viva.
\begin{itemize}
    \item Whole Blood – acute severe blood loss
    \item Packed RBC – anemia, blood loss (improves O$_2$ capacity)
    \item Fresh Frozen Plasma (FFP) – coagulation factor deficiency, DIC, massive transfusion
    \item Platelets – thrombocytopenia, platelet dysfunction, bleeding
    \item Cryoprecipitate – fibrinogen deficiency, Factor VIII, Factor XIII, vWD
    \item Albumin – hypoproteinemia, burns, shock
\end{itemize}

\subsection{Indications}
\begin{itemize}
    \item Acute blood loss $>$ 20\% blood volume
    \item Severe symptomatic anemia (Hb $<$ 7 g/dL, or $<$ 8 in cardiac disease)
    \item Coagulopathy, DIC, warfarin toxicity (FFP)
    \item Thrombocytopenia with bleeding or platelet $<$ 10–20k
    \item Hypofibrinogenemia ($<$ 100 mg/dL) – Cryoprecipitate
\end{itemize}

\subsection{Contraindications}
\begin{itemize}
    \item Not for volume replacement alone (use crystalloids/colloids)
    \item Iron-deficiency anemia without hemodynamic instability
    \item Hematinic deficiency anemia (B12/Folate) unless acute or severe
    \item Congestive Heart Failure – risk of overload (use cautiously)
\end{itemize}

\subsection{Transfusion Order (Single Unit)}
\begin{enumerate}
    \item Confirm patient ID – name, age, hospital number
    \item Blood grouping and cross-matching
    \item Check donor unit details and expiry date
    \item Bedside re-check of patient and unit
    \item Start slow rate (first 15 min), observe for reaction
    \item Monitor vitals – before, 15 min, hourly, and post-transfusion
\end{enumerate}

\subsection{Complications}
Following are the complciations of Blood Transfusion\viva

\textbf{Immediate:}
\begin{itemize}
    \item Hemolytic reaction (ABO mismatch) – fever, chills, hypotension, back pain
    \item Febrile non-hemolytic reaction
    \item Allergic – urticaria, anaphylaxis
    \item Septic shock (contaminated blood)
    \item TRALI – Transfusion-related acute lung injury
\end{itemize}

\textbf{Delayed:}
\begin{itemize}
    \item Delayed hemolysis
    \item Blood-borne infections – HIV, HBV, HCV, syphilis, malaria
    \item Iron overload
    \item Graft-vs-Host disease
\end{itemize}

\subsection{Screening of Blood Transfusion}
\begin{itemize}
    \item ABO and Rh grouping, cross-match
    \item Infectious disease screening: HIV, HBV, HCV, Syphilis, Malaria (region dependent)
\end{itemize}

\subsection{Massive Blood Transfusion}

\subsubsection{Definition}
Following are the definitions of massive Blood Transfusion\viva.
\begin{itemize}
    \item Replacement of $>$ 1 blood volume in 24 hr OR
    \item $>$ 50\% blood volume in 3 hr OR
    \item $>$ 10 units in 24 hr
\end{itemize}

\subsubsection{Precautions}
Following precautions are to be taken before Massive Blood Transfusion\viva.
\begin{itemize}
    \item Warm blood before transfusion
    \item Use screened, leucocyte-depleted components
    \item Monitor coagulation, Ca$^{2+}$, electrolytes
    \item Prefer component therapy over whole blood
\end{itemize}

\subsubsection{Complications}
Following are the complications of Massive Blood Transfusion\viva.
\begin{itemize}
    \item Dilutional coagulopathy
    \item Citrate toxicity – hypocalcemia
    \item Electrolyte imbalance – hypo/hyperkalemia
    \item Hypothermia
    \item Acid–base disturbances
\end{itemize}

\subsection{Mismatched Blood Transfusion}

\subsubsection{Features/Consequences}
\begin{itemize}
    \item Immediate fever, chills, tachycardia, hypotension
    \item Chest/back pain, hemoglobinuria, acute renal failure
    \item Shock, DIC
\end{itemize}

\subsubsection{Prevention}
\begin{itemize}
    \item Strict patient ID verification
    \item Proper grouping and cross-matching
    \item Bedside re-check before transfusion
\end{itemize}

\subsubsection{Management}
Mismatched Blood transfusion is to be managed in the following steps\viva.
\begin{itemize}
    \item Stop transfusion immediately
    \item Maintain IV access with normal saline
    \item Supportive: fluids, diuretics (forced diuresis), treat shock
    \item Monitor urine output, renal function, manage DIC if present
\end{itemize}


\section{Fluid, Electrolytes, Acid-Base Balance and Nutrition}

\section{Haemorrhage}

\section{Shock}

\section{Ulcer}

\section{Sinus and Fistula}

\section{Cyst}

\section{Swelling}

\section{Wound and Injury}

\section{Wound Debridement}

\section{Suture Materials}

\section{Scar/Keloid}

\section{Infection and Sterilization}

\section{Abscess}

\section{Carbuncle}

\section{Cellulitis}

\section{Gangrene}

\section{Tetanus}

\section{Burn}

\section{Anesthesia}

\section{Investigation and Diagnosis}

\section{Oncology}

\section{Operative Surgery}





% \begin{itemize}
%     \item \nameref{subsec:greenStickFracture}\footnotemark[1]\footnotemark[2]
%     \item Clavicular Fracture
%     \item \nameref{subsec:collesFracture}\footnotemark[1]\footnotemark[2]
% \end{itemize}
% \footnotetext[1]{Viva}
% \footnotetext[2]{Written}

% \subsection{Green Stick Fracture}
% \label{subsec:greenStickFracture}

% \subsection{Colle's Fracture}
% \label{subsec:collesFracture}
\printindex


\end{document}